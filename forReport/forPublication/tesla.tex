\NeedsTeXFormat{LaTeX2e}
\documentclass[10pt,a4paper]{book}

\usepackage{test}
\usepackage{etoolbox}
\usepackage{tikz}
\usepackage{xcolor,colortbl}
\selectlanguage{russian}
\graphicspath{{pic}}

%--- Здесь можно добавить свои команды --------
%\def\Algo500{\mbox{Algo500}}
%\def\AlgoWiki{\mbox{AlgoWiki}}
%\def\CompZoo{\mbox{CompZoo}}
%\def\CompBase{\mbox{CompBase}}
%\def\PerfData{\mbox{PerfData}}
%\def\RatingLists{\mbox{RatingLists}}


\begin{document}
    %===========================  ШАПКА СТАТЬИ =================================
    \Article{\vspace*{8mm}\par Параметрическое исследование оригинальной конструкции клапана Теслы}
    {\newpage Parametric study of the original Tesla valve design}
    
    \Abstract{В данной работе рассматривалась оригинальная конструкция гидродинамического диода в котором основным исследуемым параметром был выбран угол наклона отводящего канала. Численно исследован стационарный режим течения несжимаемой линейно-вязкой жидкости. Получена зависимость диодности устройства от угла наклона отводящего канала. Рассмотрен вопрос сеточной сходимости решения. Выбраны оптимальные параметры расчетной сетки.}
    {In this paper, we considered the original design of a hydrodynamic diode in which the main parameter under study was the angle of inclination of the discharge channel. The stationary flow regime of an incompressible linearly viscous liquid is numerically investigated. The dependence of the diode of the device on the angle of inclination of the discharge channel is obtained. The issue of grid convergence of the solution is considered. The optimal parameters of the calculated grid are selected.}
    
    
    \Keywords{Численные методы, параметрическая модель, клапан Тесла, сеточная сходимость, параметрическое исследование, стационарный режим, OpenFoam.}
    {Numerical methods, parametric model, Tesla valve, mesh convergence, parametric investigation, stationary mode, OpenFOAM.}
    
    \Acknowledgements{слова благодарности ...}
    {eng text}
    
    \Citation{Бондаренко Н. А., ...
        Параметрическое исследование оригинальной конструкции клапана Теслы//
        Вычислительные методы и программирование. 2024.
        \textbf{??}, \No~?. \pageref*{firstPage}--\pageref*{LastPage}.  
        doi 10.26089/NumMet.v??r???.}
    {Bondarenko N. A., ...
        Parametric study of the original Tesla valve design//
        Numerical Methods and Programming. 2024.
        \textbf{??}, \No~?. \pageref*{firstPage}--\pageref*{LastPage}.  
        doi 10.26089/NumMet.v??r???.}
    
    
    \UDC{517.968;\\  519.642.3}
    \DOI{10.26089/NumMet.v??r???}
    \VOL{??}{?}
    \YEAR{2024}
    \Received{день месяц 2024 г.}{m. d., 2024}
    \Accepted{день месяц 202? г.}{eng text}
    
    
    \Author{Н. А. Бондаренко}{Nikita A.~Bondarenko}
    \FullName{Бондаренко Никита Александрович}{Nikita A.~Bondarenko}
    \Institution{Тюменкский государственный университет,\break 
        Laboratory for the study of microfiltration processes}
    {University of Tyumen, \break
        }
    \Address{Ленина, 25}{Lenin street, 25}
    \Postcode{625003}
    \City{Тюмень}{Tyumen}
    \CountryOfResidence{Российская Федерация}{Russia}
    \AcademicDegree{Магистратура, 2 курс}{Master's degree, 2nd year} %?????
    \Position{Лаборант исследователь}{Laboratory researcher}
    \Orcid{????-????-????-????}
    \Email{bndnikita@gmail.com}
    
%    ?????
    %\Author{П.~П.~Петров}{Petr P.~Petrov}
    %\FullName{Петр Петрович Петров}{Petr P.~Petrov}
    %\Institution{Московский государственный университет имени М. В. Ломоносова,\break 
    %    Научно-исследовательский вычислительный центр}
    %{Lomonosov Moscow State University, Research Computing Center}
    %\Address{Ленинские горы, 1, стр.~4}{Leninskie Gory, 1, building 4}
    %\Postcode{119234}
    %\City{Москва}{Moscow}
    %\CountryOfResidence{Российская Федерация}{Russia}
    %\AcademicDegree{}{}
    %\Position{программист}{programmer}
    %\Orcid{????-????-????-????}
    %\Email{petrov@gmail.com}
    
    \MakeArticleHeader
    %========================= КОНЕЦ ШАПКИ СТАТЬИ ==============================
    
    \section{Введение}\label{p:vved}
    Характерной особенностью гидродинамических диодов, представителем которых является клапан Тесла~[\ref{TeslaValveReview}], является то, что при заданном расходе флюида необходимо прикладывать разный перепад давления в зависимости от направления потока. Построение геометрии каналов по выбранному шаблону позволяет генерировать сложные схемы течения при обратном подключении, когда перепад давления наибольший, и более плавные схемы при прямом подключении, когда перепад наименьший. Данные устройства нашли широкое применение в науке и технике поскольку сочетают в себе простоту конструкции за счет отсутствия механических подвижных частей и эффективность. Клапаны такого типа обладают такими преимуществами как масштабируемость, долговечность и простота изготовления из различных материалов.
    
    В статье~[\ref{JIN20188888}] было проведено численное исследование работы клапана Теслы в качестве декомпрессора для водородного топлива при заправке электромобиля. Скорость, с которой сжатый водород выходит слишком высокая и повреждает топливную систему, однако, Клапан Тесла способен замедлить ток флюида поступающего в автомобиль. Анализируется распределение давления в клапане при различных скоростях. Высокий перепад давления достигается при малом гидравлическом диаметре, большом угле наклона и малом внутреннем радиусе отводного канала. 
    
    В работе~[\ref{DU2023103670}] представлена оригинальная конфигурация клапана Теслы со встроенным миксером. Также продемонстрирована модель прогнозирования и оптимизации тепловых и гидравлических характеристик гидродинамического диода. Исследователи изучают влияние внутреннего радиуса отводящего канала, длины U-образного сегмента и угла наклона отводящего канала на число Нуссельта и перепад давления.
    
    В данной работе~[\ref{article}] было исследовано применение микрофлюидных устройств в медицине. Был представлен панорамный обзор технологий и материалов, которые могут быть использованы для лечения Гидроцефалии. Представлена наиболее подходящая комбинация материалов, технологий и конструкций для получения полностью имплантируемого микронасоса. 
    
    В статье~[\ref{SONG2014407}] проводился анализ предохранительного клапана прямого действия, установленного на сосуде высокого давления. Была разработана численная модель для исследования текучих и динамических характеристик клапана. Работа клапана моделируется с использованием трехмерной подвижной сетки.
    
    В работе~[\ref{OptimizationOfTheFixed-GeometryValve}] рассматривался клапанный микронасос с фиксированной геометрией. Был предоставлен процесс для оптимизации формы клапана с использованием вычислительной гидродинамики в двумерной постановке в сочетании с алгоритмом оптимизации. Диодность клапана была улучшена за счет манипулирования шестью безразмерными, независимыми геометрическими параметрами клапана Теслы.
    
    В исследовании~[\ref{inproceedings}] представлено параметрическое описание двух типов клапанов с фиксированной геометрией, сопло-диффузор и клапан Тесла. Был создан новый тип клапана -- клапан Тессер, обладающий характеристиками как клапана Теслы, так и сопла-диффузора. Были проведены расчеты полученного клапана в двумерной постановке. 
    
    Перейдем к сути проделанного исследования. В работе представлена параметрическая модель геометрии клапана Теслы, реализованная при помощи Salome Python API. В результате исследования было установлено оптимальное разрешение расчетной сетки. Параметрическое исследование выявило, при каком угле между отводящем каналом и основным клапан Тесла имеет наибольшее отношение между перепадами давления при обратном и прямом подключении.
    
    \section{Геометрия расчетной области и расчетная сетка}\label{p:roles}
    
    В настоящее время существует множество различных вариаций клапана Теслы, заметно отличающихся от оригинальной версии, представленной в патенте US1329559A самим Николо Тесла. В нашем исследовании используется следующая параметризованная модель геометрии клапана Теслы:
    
    \begin{figure}[H]
        \centering
        \includegraphics[width = 1\linewidth]{teslaDesign}
        \caption{Клапан Тесла.}
        \label{fig:td}
    \end{figure}
    
    Это прямой канал с U-образным отводным каналом. Задача отводного канала, разделить ток жидкости идущей по основному каналу, и перенаправить отведенную часть против основного потока. Таким образом, клапан Тесла, при обратном подключении и работает, замедляя ток жидкости. Заметным отличием от оригинальной версии клапана Теслы является явно выделенный основной канал, к которому добавлены отводящие каналы.          
    
    На основе выбранного шаблона был написан скрипт на языке Python, для построения параметрической геометрии и расчетной сетки с минимальным количеством независимых параметров. На рис.~\ref{fig:td} схематично показана зависимость геометрии клапана от вводимых параметров, всего независимых параметров два - это ширина канала, W, и угол наклона отводящего канала, $ \alpha  $. Скругление на входе в отводящий канал влияет на эффективность клапана, увеличивая площадь входа в U - образный канал. Основной параметр для параметрического исследования -- это угол $ \alpha  $, а для исследования сеточной сходимости -- это максимальный характерный размер одной ячейки сетки, в микрометрах, при сохранении отношения между минимальным и максимальным размером ячеек.
    
    На рис.~\ref{fig:teslaMesh2} угол $\alpha$ равен 30 градусам, ширина канала была выбрана равной 500 мкм или 0.0005 м. Также, можно видеть, что сетка не однородная. В нашей задаче основной интерес сосредоточен в областях, где поток разделяется и перемешивается. В них могут происходить отрывы или скачки, для разрешения градиентов давления и более точного определения профиля скорости разрешение сетки в этих областях выше.
    
    \begin{figure}[H]
        \centering
        \includegraphics[width = 0.7\linewidth]{meshWithZoom}
        \caption{Расчетная сетка.}
        \label{fig:teslaMesh2}
    \end{figure}   
    
    Для разрешения градиента скорости вблизи стенок был добавлен сеточный подслой. Скругления острых углов сделано с целью построения более качественной сетки, радиус скругления фиксирован.
    
    \section{Режим течения и граничные условия}        
    
    Чтобы определить какого рода перед нами течение, можно рассчитать число Рейнольдса, Re. Исходя из полученного значения, можно будет сделать выводы о характере потока, турбулентное или ламинарное.
    Так как канал нашей конфигурации клапана Теслы имеет квадратное сечение, то формула для определения числа Рейнольдса имеет вид:
    
    \begin{equation}\label{eqn:Re}
        Re = \frac{u D_{H}}{\nu},
    \end{equation}            
    где  u - скорость в канале, м/с, $ D_{H} = \frac{4A}{P} $ - гидравлический диаметр, м, $\nu$ - кинематическая вязкость, м$^{2}/$с. 
    где A - площадь поперечного сечения канала, м$^{2}$, P - смоченный периметр. 
    
    Для нашей конфигурации была выбрана скорость, задаваемая на входе, равной 3 м/с, а число соответственно равно Рейнольдса - 1500.
    
    OpenFoam - это открытый программный комплекс для решения задач механики сплошной среды. SimpleFoam - это стационарный решатель для несжимаемого турбулентного потока, использующий алгоритм SIMPLE. Математическая модель, реализованная в решателе SimpleFoam, имеет вид:
    
    \begin{equation}\label{eqn:simpleFoam}
        \bm{\nabla} \cdot \bm{u} = 0
    \end{equation} 
    
    \begin{equation}\label{eqn:simpleFoam2}
        \bm{\nabla} \cdot \bm{u} \otimes \bm{u} = -\bm{\nabla} p + \bm{\nabla} \cdot \bm{\tau}
    \end{equation} 
    
    Где $\bm{u}$ - скорость, м/с, p - кинематическое давление, м$^{2}/$с$^{2}$, $\bm{\tau}$ - тензор напряжения. 
    Каждый цикл итерации влечет за собой сначала расчет промежуточного поля скорости, которое удовлетворяет линеаризованным уравнениям импульса для предполагаемого распределения давления: затем применяется принцип сохранения массы для настройки скоростей и давлений, так что все уравнения находятся в равновесии.
    
    Выбранным типом моделирования турбулентности был параметр RAS. Модель k-epsilon объединяет уравнения турбулентной кинетической энергии (\ref{eqn:k}), k, и уравнение скорости рассеивания турбулентной кинетической энергии (\ref{eqn:e}), $\epsilon$.
    
    \begin{equation}\label{eqn:k}
        \frac{D}{D_{t}}(\rho k) = \nabla \cdot (\rho D_{k}\nabla k) + P - \rho\epsilon
    \end{equation} 
    
    \begin{equation}\label{eqn:e}
        \frac{D}{D_{t}}(\rho\epsilon) = \nabla \cdot (\rho D_{\epsilon}\nabla\epsilon) + \frac{C_{1}\epsilon}{k}(P + C_{3}\frac{2}{3}k\nabla \cdot \bm{u}) - C_{2}\rho\frac{\epsilon^2}{k}
    \end{equation} 
    Где k --- турбулентная кинетическая энергия, м$^2$/с$^2$, $D_{k}$ - Эффективная диффузионная способность для k, P - скорость производства турбулентной кинетической энергии, м$^2$/с$^-3$, $\epsilon$ - скорость рассеивания турбулентной кинетической энергии, м$^2$/с$^-3$, $D_{\epsilon}$ - Эффективная диффузионная способность для $\epsilon$.
    
    Далее решается уравнение для турбулентной вязкости:
    
    \begin{equation}\label{eqn:mu}
        \nu_{t} = C_{\mu}\frac{k^2}{\epsilon}
    \end{equation} 
    Где $C_{\mu}$ - модельный коэффициент турбулентной вязкости, $\mu_{t}$ - турбулентная вязкость, м$^2$/с$^-1$.
    
    Граничные условия для решения уравнений турбулентности заданы через интенсивность для k (\ref{eqn:kI}) и через длину перемешивания для $\epsilon$ (\ref{eqn:el}). Граничные условия для скорости заданы через объемный расход, для давления через абсолютное давление (\ref{eqn:P0}).
    
    \begin{equation}\label{eqn:kI}
        k_{p} = 1.5 (I |U|)^2
    \end{equation}
    Где $k_{p}$ - кинетическая энергия на границе, м$^2$/с$^2$, I - интенсивность турбулентности.
    
    \begin{equation}\label{eqn:el}
        \epsilon_{p} = \frac{C_{\mu}^{0.75} k^{1.5}}{L}           
    \end{equation}
    Где $\epsilon_{p}$ - диссипация кинетической энергии на границе, м$^2$/с$^-3$, L - шкала длины.
    
    \begin{equation}\label{eqn:P0}
        p_{p} = p_{0} + \frac{1}{2}\ \left|u_{0}\right|^2 - \frac{1}{2}\ \left|u\right|^2
    \end{equation}
    Где $p_{p}$ - давление на границе, м$^{2}/$с$^{2}$, $p_{0}$ - внешнее статическое давление, м$^{2}/$с$^{2}$, $u$ - скорость, м/с, $u_{0}$ - внешняя скорость, м/с.\\
    
    Оценить эффективность клапана Теслы, после получения результатов расчета, мы можем, посчитав его диодность, Di (\ref{eqn:Di}). Если Di > 1, то рассматриваемый клапан можно считать рабочим. Для этого мы проводили расчет нашей конфигурации клапана Тесла с одинаковыми параметрами дважды, но при разных подключениях: при обратном, когда перепад давления наибольший, и, при прямом, когда перепад давления наименьший. Полученные данные фиксировались.         
    
    \begin{equation}\label{eqn:Di}
        Di = (\frac{\bigtriangleup p_{r}}{\bigtriangleup p_{f}})_Q
    \end{equation}
    Где $\bigtriangleup p_{r}$ - перепад давления при обратном подключении, $\bigtriangleup p_{f}$ - перепад давления при прямом подключении для скорости потока Q.
    
    \section{Ход исследования}
    
    При выборе минимального разрешения сетки, мы остановились на таком разрешении, которое позволяло бы на входе поместиться 5 ребрам ячеек расчетной сетки (рис.~\ref{fig:minMesh}). Такое решение было принято в пользу лучшей сходимости задачи.
    
    \begin{figure}[H]
        \centering
        \includegraphics[width = 1\linewidth]{minMesh}
        \caption{Минимальное разрешение сетки.}
        \label{fig:minMesh}
    \end{figure}
    
    Рассмотрим полученные в ходе расчетов поля давления и скорости при обратном подключении (рис.~\ref{fig:UPFieldsReverse}). На изображении поля давления видно, как по мере удаления от входа, давление в клапане снижается и имеет вид приближенный к линейному. Видно, как в местах где поток разделяется и смешивается, происходят скачки давления. Изображение поля скорости, демонстрирует нам значительное падание скорости между участками клапана, где происходит разделение и смешивание потоков жидкости.
    
    \begin{figure}[H]
        \centering
        \includegraphics[width = 1\linewidth]{UPFieldsReverse}
        \caption{Поле скорости и давления при обратном подключении.}
        \label{fig:UPFieldsReverse}
    \end{figure}
    
    Поле скорости (рис.~\ref{fig:UPFieldsDirect}) при прямом подключении. Видно, как в клапане образуется явно выделенное ядро потока, течение более не рассеивается геометрией клапана так, как это было при обратном подключении. Поле давления линейно, и можно видеть, как отдаляясь от входа клапана, давление падает.
    
    \begin{figure}[H]
        \centering
        \includegraphics[width = 1\linewidth]{UPFieldsDirect}
        \caption{Поле скорости и давления при прямом подключении.}
        \label{fig:UPFieldsDirect}
    \end{figure}   
    
    \section{Исследование сеточной сходимости}
    Было принято решение отталкиваться от максимального размера ячейки. Каждая следующая сетка отличалась от предыдущей тем, что максимальный размер ячейки увеличивается на 20\%. График зависимости диодности клапана Тесла от разрешения сетки, где мерой разрешения является количество элементов сетки.

    \begin{figure}[H]
            \centering
            \includegraphics[width = 1\linewidth]{allGraph}
            \caption{Результаты исследования. $a$ - зависимость диодности от количества элементов, $b$ - дополнительные расчеты зависимости диодности от количества элементов с добавлением линий тренда, $c$ - зависимость диодности от максимального размера ячейки расчетной сетки, $d$ - зависимость диодности от угла $\alpha$.}
            \label{fig:allGraph}
        \end{figure}
    
    По графику $a$ (рис.~\ref{fig:allGraph}) можно видеть, что с увеличением количества элементов сетки растет и диодность, но этого графика недостаточно для определения оптимального разрешения сетки. Для того, чтобы сделать вывод о сеточной сходимости, нам надо выполнить ряд дополнительных расчетов и выяснить, после какого разрешения диодность замедлит свой рост и последующее увеличении плотности расчетной сетки не будет давать значительного прироста точности. 
    
    По представленным на графике $b$ (рис.~\ref{fig:allGraph}) данным можно видеть, как с последующем увеличением разрешения сетки, диодность резко возрастает, однако, в последствии, рост диодности заметно замедляется. Исходя из этого, можно сделать вывод о сеточной сходимости. Достигая определенной плотности сетки, дальнейшее увеличение количества элементов не приведет к значительному улучшению решения, но потребует дополнительных вычислительных ресурсов. Добавленные линии тренда позволяют нам увидеть, что сетки до 5 млн элементов имеют рост диодности выше, чем сетки с более высокой плотностью, но среднее значение диодности у них меньше. Средний рост диодности для сеток с разрешением менее 5 миллионов элементов составляет 1,75 \textdiscount, в то время как для сеток с большим разрешением средний рост диодности составляет уже 0,99 \textdiscount. Если взять максимальное разрешение сетки за эталонное, то есть максимально приближенное к правильному решению, и посчитать во сколько процентов отличаются решения с меньшем разрешением, то результаты, полученные на сетках с разрешением менее 5 млн. элементов отличаются в более чем на 5 \textdiscount. Для дальнейшего параметрического исследования, нам следует выбрать сетку, плотность которой можно считать оптимальной. Для этого построим график зависимости диодности от максимального размера ячейки сетки.
    
    Из графика $c$ (рис.~\ref{fig:allGraph}) можно сделать вывод о том, что значительный прирост диодности происходит при максимальном характерном размере ячейки сетки равным 40 мкм. Этот размер мы и будем использовать при параметрическом исследовании. Что позволить нам получать более точные результаты оптимально задействовав вычислительные мощности. 

    \section{Параметрическое исследование клапана Теслы}
    В результате проведения ряда расчетов, используя оптимальную плотность сетки определенную в ходе исследования сеточной сходимости. В этих расчетах мы изменяли угол $ \alpha $ (рис.~\ref{fig:td}), определяя угол при котором диодность будет максимальной. Минимальное значение для угла $ \alpha $ равно 10\textdegree, а максимальное 70\textdegree. В ходе исследования, шаг для угла был взять равным 5\textdegree. В месте, с предполагаемым пиком диодности, 20\textdegree, шаг был равным 1,25\textdegree. 
    
    \begin{figure}[H]
        \centering
        \includegraphics[width = 1\linewidth]{allAngle1}
        \caption{Построение геометрии при разном угле $\alpha$.}
        \label{fig:allAngle}
    \end{figure}  
          
    Видно, как изменение угла $\alpha$ влияет на общую длину основного и отводящего каналов (рис.~\ref{fig:allAngle}). Это значит, что при постоянной плотности сетки, количество её ячеек, как и требуемые вычислительные ресурсы, будет расти при уменьшении угла между основным и отводящим каналом. В результате параметрического исследования мы получаем следующий график зависимости диодности от угла $ \alpha $.
  
    В результате параметрического исследования (рис.~\ref{fig:allGraph}, график $c$) было выяснено, что наша конфигурация клапана Теслы имеет максимальную диодность при угле $ \alpha $ равным 17,5 градуса. 
    
    \section{Выводы}
    
    Исследование сеточной сходимости показало, что при решении задач численными методами, важно соблюдать баланс между точностью решения и трат вычислительных ресурсов. Высокая точность требует больших затрат по времени и высокой производительности вычислительной техники. При проведении параметрического исследования нашей конфигурации клапана Теслы, был проведен ряд расчетов позволяющих резюмировать, что, при угле $\alpha$  = 17,5\textdegree, клапан Тесла нашей конфигурации имеет эффективность своей работы оцениваемую в $Di$ = 1,431.
    
    
    \vspace*{10mm plus 2mm minus 2mm}
    
    %=======================  Список литературы ==================================
    \LITERRUS
    
    %---------------------------------------------
    %Моя литература на русском (начало)
    %---------------------------------------------
    
    %1
    \elitem{TeslaValveReview}{%1
    Purwidyantri, A.;Prabowo, B.A.
    ``Tesla Valve Microfluidics: The Rise of ForgottenTechnology,''
    Chemosensors 2023. {\bf 11}, 256.
    \doi{10.3390/chemosensors11040256}.
    }
    
    %2
    \elitem{JIN20188888}{%2
        Zhi-jiang Jin, Zhi-xin Gao, Min-rui Chen, Jin-yuan Qian,
        Parametric study on Tesla valve with reverse flow for hydrogen decompression,
        International Journal of Hydrogen Energy,
        Volume 43, Issue 18,
        2018,
        Pages 8888-8896,
        ISSN 0360-3199.
        \doi{10.1016/j.ijhydene.2018.03.014}.
    }
    
    %3
    \elitem{DU2023103670}{%3
        Gang Du, Theyab R. Alsenani, Jitendra Kumar, Salem Alkhalaf, Tamim Alkhalifah, Fahad Alturise, Hamad Almujibah, Sami Znaidia, Ahmed Deifalla,
        Improving thermal and hydraulic performances through artificial neural networks: An optimization approach for Tesla valve geometrical parameters,
        Case Studies in Thermal Engineering,
        Volume 52,
        2023,
        103670,
        ISSN 2214-157X.
        \doi{10.1016/j.csite.2023.103670}.
    }
    
    %4
    \elitem{article}{%4
        Morganti, Elisa and Pignatel, Giorgio. (2005). Microfluidics for the treatment of the hydrocephalus. Proc. 1st Int. Conf. on Sensing Technology. 
    }
    
    %5
    \elitem{SONG2014407}{%5
        Xueguan Song, Lei Cui, Maosen Cao, Wenping Cao, Youngchul Park, William M. Dempster,
        A CFD analysis of the dynamics of a direct-operated safety relief valve mounted on a pressure vessel,
        Energy Conversion and Management,
        Volume 81,
        2014,
        Pages 407-419,
        ISSN 0196-8904,
        \doi{10.1016/j.enconman.2014.02.021}.
    }
    
    %6
    \elitem{OptimizationOfTheFixed-GeometryValve}{%6
        Gamboa, A. R., Morris, C. J., and Forster, F. K. (2003). Optimization of the Fixed-Geometry Valve for Increased Micropump Performance. Fluids Engineering. doi:10.1115/imece2003-55036 
        \doi{10.1115/imece2003-55036}.
    }
    
    %7
    \elitem{inproceedings}{%7
        Forster, Fred and Williams, Brian. (2002). Parametric Design of Fixed-Geometry Microvalves: The Tesser Valve. ASME International Mechanical Engineering Congress and Exposition, Proceedings. 
        \doi{10.1115/IMECE2002-33628}.
    }
    
%    %8
%    \elitem{article22}{%8
%        А. А. Гаврилов, А. А. Дектерев, А. В. Шебелев. Простой критерий оценки сеточной детализации для RANS методов. Журнал вычислительной математики и математической физики, 2023, T. 63, № 4, стр. 533-547 
%        \doi{10.31857/S0044466923040087}.
%    }
    
    
    %---------------------------------------------
    %Моя литература на русском (конец)
    %---------------------------------------------
    
    %=======================  Список литературы ======================
    \medskip
    
    \DateRU  
    
    \bigskip
    \bigskip
    
    \MakeAuthorsInfoRU
    
    
    %=========================== REFERENCES =========================
    \bigskip
    \bigskip
    
    
    \REFERENCES
    
    %---------------------------------------------
    %Моя литература на английском (начало)
    %---------------------------------------------
    
    %1
    \elitem{e:TeslaValveReview}{%1
        Purwidyantri, A.;Prabowo, B.A.
        ``Tesla Valve Microfluidics: The Rise of ForgottenTechnology,''
        Chemosensors 2023. {\bf 11}, 256.
        \doi{10.3390/chemosensors11040256}.
    }
    
    %2
    \elitem{e:JIN20188888}{%2
        Zhi-jiang Jin, Zhi-xin Gao, Min-rui Chen, Jin-yuan Qian,
        Parametric study on Tesla valve with reverse flow for hydrogen decompression,
        International Journal of Hydrogen Energy,
        Volume 43, Issue 18,
        2018,
        Pages 8888-8896,
        ISSN 0360-3199.
        \doi{10.1016/j.ijhydene.2018.03.014}.
    }
    
    %3
    \elitem{e:DU2023103670}{%3
        Gang Du, Theyab R. Alsenani, Jitendra Kumar, Salem Alkhalaf, Tamim Alkhalifah, Fahad Alturise, Hamad Almujibah, Sami Znaidia, Ahmed Deifalla,
        Improving thermal and hydraulic performances through artificial neural networks: An optimization approach for Tesla valve geometrical parameters,
        Case Studies in Thermal Engineering,
        Volume 52,
        2023,
        103670,
        ISSN 2214-157X.
        \doi{10.1016/j.csite.2023.103670}.
    }
    
    %4
    \elitem{e:article}{%4
        Morganti, Elisa and Pignatel, Giorgio. (2005). Microfluidics for the treatment of the hydrocephalus. Proc. 1st Int. Conf. on Sensing Technology. 
    }
    
    %5
    \elitem{e:SONG2014407}{%5
        Xueguan Song, Lei Cui, Maosen Cao, Wenping Cao, Youngchul Park, William M. Dempster,
        A CFD analysis of the dynamics of a direct-operated safety relief valve mounted on a pressure vessel,
        Energy Conversion and Management,
        Volume 81,
        2014,
        Pages 407-419,
        ISSN 0196-8904,
        \doi{10.1016/j.enconman.2014.02.021}.
    }
    
    %6
    \elitem{e:OptimizationOfTheFixed-GeometryValve}{%6
        Gamboa, A. R., Morris, C. J., and Forster, F. K. (2003). Optimization of the Fixed-Geometry Valve for Increased Micropump Performance. Fluids Engineering. doi:10.1115/imece2003-55036 
        \doi{10.1115/imece2003-55036}.
    }
    
    %7
    \elitem{e:inproceedings}{%7
        Forster, Fred and Williams, Brian. (2002). Parametric Design of Fixed-Geometry Microvalves: The Tesser Valve. ASME International Mechanical Engineering Congress and Exposition, Proceedings. 
        \doi{10.1115/IMECE2002-33628}.
    }
    
%    %8
%    \elitem{e:article22}{%8
%        A. A. Gavrilov, A. A. Dekterev, A. V. Shebelev. A simple criterion for evaluating grid detail for RANS methods. Journal of Computational Mathematics and Mathematical Physics, 2023, V. 63, № 4, pages 533-547 
%        \doi{10.31857/S0044466923040087}.
%    }
         
    %---------------------------------------------
    %Моя литература на английском (конец)
    %---------------------------------------------
    
    \medskip
    \DateEN
    
    \bigskip
    \bigskip
    \bigskip
    
    \MakeAuthorsInfoEN
    
    %============================ КОНЕЦ СТАТЬИ ==================================
    \end{document}