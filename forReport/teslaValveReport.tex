\documentclass[14pt,a4paper]{report}

\usepackage[utf8]{inputenc}
\usepackage[english,russian]{babel}
\usepackage[T2A]{fontenc}
\usepackage{graphicx}
\usepackage{amssymb}
\usepackage{amsthm}
\usepackage{bm}
\graphicspath{{media}}
\usepackage{float}
\usepackage{}
%\setlength{\parindent}{1.25cm}
\usepackage{indentfirst}

\begin{document}
    \section*{Введение.}
    
    
    
    \section*{Параметрическая геометрия.}
        
        Для дальнейшего исследования был выбран следующий шаблон геометрии клапана Теслы:
        
        \begin{figure}[h!]
            \centering
            \includegraphics[width=100mm,scale=0.5]{teslaValve}
            \caption{Клапан Тесла.}
            \label{fig:TeslaValve}
        \end{figure}
        Это прямой канал с U-образным отводным каналом.
        Основной параметр для дальнейшего исследования -- это угол $ \alpha  $.
        
        На основе выбранного шаблона был написан скрипт для построения параметрической геометрии и расчетной сетки.
        
        \begin{figure}[h!]
            \centering
            \includegraphics[width = 1\linewidth]{teslaMesh1}
            \caption{Расчетная сетка.}
            \label{fig:teslaMesh}
        \end{figure}
        
        На (рис.~\ref{fig:teslaMesh}) угол $\alpha$ равен 30 градусам. Также, можно видеть, что сетка не однородная и ее разрешение падает по мере удаления от зон с наибольшим интересом.
        
        \begin{figure}[H]
            \centering
            \includegraphics[width = 1\linewidth]{teslaMesh2}
            \caption{Расчетная сетка в приближении.}
            \label{fig:teslaMesh2}
        \end{figure}   
             
        Для разрешения градиента скорости вблизи стенок был добавлен сеточный подслой (рис.~\ref{fig:teslaMesh2}).
        
    \section*{Режим течения и расчет.}        
        
        Чтобы определить какого рода перед нами течение, можно рассчитать число Рейнольдса, Re. Исходя из полученного значения, можно будет сделать выводы о характере потока, турбулентное или ламинарное.
        Так как канал нашей конфигурации клапана Теслы имеет квадратное сечение, то формула для определения числа Рейнольдса имеет вид:
        
        \begin{equation}\label{eqn:Re}
            Re = \frac{u D_{H}}{\nu}
        \end{equation}
            
        Где  u - скорость в канале, м/с, $ D_{H} $ - гидравлический диаметр, м, $\nu$ - кинематическая вязкость, м$^{2}/$с. 
        
        $D_{H} = \frac{4A}{P}$,\\
        где A - площадь поперечного сечения канала, м$^{2}$, P - смоченный периметр. 
        
        Для нашей конфигурации скорость течения в канале будет равна 3 м/с, а число Рейнольдса - 1500. Ширина канала была выбрана равной 500 мкм или 0.0005 м.  
        
        Для числа Рейнольдса    
        
        OpenFoam - это открытый программный комплекс для решения задач механики сплошной среды. SimpleFoam - это стационарный решатель для несжимаемого турбулентного потока, использующий алгоритм SIMPLE. Математическая модель, реализованная в решателе simpleFoam, имеет вид:
        
        \begin{equation}\label{eqn:simpleFoam}
            \bm{\nabla} \cdot \bm{u} = 0
        \end{equation} 
        
        \begin{equation}\label{eqn:simpleFoam2}
            \bm{\nabla} \cdot \bm{u} \otimes \bm{u} = -\bm{\nabla} p + \bm{\nabla} \cdot \bm{\tau}
        \end{equation} 
        
        Где $\bm{u}$ - скорость, м/с, p - кинематическое давление, м$^{2}/$с$^{2}$, $\bm{\tau}$ - тензор напряжения. 
        Каждый цикл итерации влечет за собой сначала расчет промежуточного поля скорости, которое удовлетворяет линеаризованным уравнениям импульса для предполагаемого распределения давления: затем применяется принцип сохранения массы для настройки скоростей и давлений, так что все уравнения находятся в равновесии.
        
       По графикам невязок (два скрина невязок.) видно, что при решении без использования модели турбулентности, решение является неустойчивым. Исходя из этого было принято решение о подключении турбулентной модели k-epsilon. Выбранным типом моделирования турбулентности был параметр RAS. Модель k-epsilon объединяет уравнения турбулентной кинетической энергии (\ref{eqn:k}), k, и уравнение скорости рассеивания турбулентной кинетической энергии (\ref{eqn:e}), $\epsilon$.
        
        \begin{equation}\label{eqn:k}
            \frac{D}{D_{t}}(\rho k) = \nabla \cdot (\rho D_{k}\nabla k) + P - \rho\epsilon
        \end{equation} 
        
        \begin{equation}\label{eqn:e}
            \frac{D}{D_{t}}(\rho\epsilon) = \nabla \cdot (\rho D_{\epsilon}\nabla\epsilon) + \frac{C_{1}\epsilon}{k}(P + C_{3}\frac{2}{3}k\nabla \cdot \bm{u}) - C_{2}\rho\frac{\epsilon^2}{k}
        \end{equation} 
       Где k --- турбулентная кинетическая энергия, м$^2$/с$^2$, $D_{k}$ - Эффективная диффузионная способность для k, P - скорость производства турбулентной кинетической энергии, м$^2$/с$^-3$, $\epsilon$ - скорость рассеивания турбулентной кинетической энергии, м$^2$/с$^-3$, $D_{\epsilon}$ - Эффективная диффузионная способность для $\epsilon$.
               
        Далее решается уравнение для турбулентной вязкости:
        
        \begin{equation}\label{eqn:mu}
           \nu_{t} = C_{\mu}\frac{k^2}{\epsilon}
        \end{equation} 
        Где $C_{\mu}$ - модельный коэффициент турбулентной вязкости, $\mu_{t}$ - турбулентная вязкость, м$^2$/с$^-1$.
        
        Граничные условия. Для решения уравнений турбулентности заданы через интенсивность для k (\ref{eqn:kI}) и через длину перемешивания для $\epsilon$ (\ref{eqn:el}). Граничные условия для скорости заданы через объемный расход, для давления через абсолютное давление (\ref{eqn:P0}).
        
        \begin{equation}\label{eqn:kI}
            k_{p} = 1.5 (I |U|)^2
        \end{equation}
        Где $k_{p}$ - кинетическая энергия на границе, м$^2$/с$^2$, I - интенсивность турбулентности.
        
        \begin{equation}\label{eqn:el}
            \epsilon_{p} = \frac{C_{\mu}^{0.75} k^{1.5}}{L}           
        \end{equation}
        Где $\epsilon_{p}$ - диссипация кинетической энергии на границе, м$^2$/с$^-3$, L - шкала длины.
        
        \begin{equation}\label{eqn:P0}
            p_{p} = p_{0} + \frac{1}{2}\ \left|u_{0}\right|^2 - \frac{1}{2}\ \left|u\right|^2
        \end{equation}
        Где $p_{p}$ - давление на границе, м$^{2}/$с$^{2}$, $p_{0}$ - внешнее статическое давление, м$^{2}/$с$^{2}$, $u$ - скорость, м/с, $u_{0}$ - внешняя скорость, м/с.
        
        Оценка полученных результатов. Для оценки эффективности клапана Тесла, оценим его диодность, Di (\ref{eqn:Di}). Если Di > 1, то рассматриваемый клапан можно считать рабочим. Для этого мы проводили расчет нашей конфигурации клапана Тесла с одинаковыми параметрами дважды, но при разных подключениях: при обратном, когда перепад давления наибольший, и, при прямом, когда перепад давления наименьший. Полученные данные фиксировались.         
        
        \begin{equation}\label{eqn:Di}
            Di = (\frac{\bigtriangleup p_{r}}{\bigtriangleup p_{f}})_Q
        \end{equation}
        Где $\bigtriangleup p_{r}$ - перепад давления при обратном подключении, $\bigtriangleup p_{f}$ - перепад давления при прямом подключении для скорости потока Q.
        
        \section*{Результаты.}
        
        
    
\end{document}    